\documentclass[12pt,a4]{article}



\input{preamble}



\setcounter{section}{1}


\begin{itemize}
 \item Homework assignment published on Monday, 2018-03-05.
 \item Work on it and submit a first solution or questions by Sunday, 2018-03-11, 12:00 by
 email to me and the TAs.
 \item You will receive feedback by Wednesday, 2018-03-14.
 \item Submit your final solution by Sunday, 2018-03-18 to me and the TAs.
\end{itemize}



\section{Partial Orderings}



\subsection{Equivalence Relations as a Partial Ordering}

An equivalence relation $R \subseteq V \times V$ is basically the same as a partition
of $V$. A {\em partition} of $V$ is a set $\{V_1,\dots,V_k\}$ where 
(1) $V_1 \cup \dots \cup V_k = V$ and (2) the $V_i$ are pairwise disjoint,
i.e., $V_i \cap V_j  = \emptyset$ for $1 \leq i < j \leq k$. For example,
$\{ \{1\}, \{2,3\}, \{4\} \}$ is a partition of $\{1,2,3,4\}$ but 
$\{ \{1\}, \{2,3\}, \{1,4\}\}$ is not. 



\begin{exercise}
  Let $E_4$ be the set of all equivalence relations on $\{1,2,3,4\}$. Note that
  $E_4$ is ordered by set inclusion, i.e., 
  \begin{align*}
     (E_4, \{ (R_1,R_2) \in E_4 \times E_4 \ | \ R_1 \subseteq R_2 \} )
  \end{align*}
  is a partial ordering. 
  \begin{enumerate}
    \item Draw the Hasse diagram of this partial ordering in a  nice way.
    \item What is the size of the largest chain?
    \item What is the size of the largest antichain? 
  \end{enumerate}
  \end{exercise}


\subsection{Chains and Antichains}

Define the partially ordered set $(\nn, \leq)$ as follows:
$x \leq y$ if $x_i \leq y_i$ for all $1 \leq i \leq n$. For example,
$(2,5,4) \leq (2,6,6)$ but $(2,5,4) \not \leq (3,1,1)$.

\begin{exercise}
  Consider the infinite partially ordered set $(\nn, \leq)$.
  \begin{enumerate}
  \item
    Which elements are minimal? Which are maximal?
  \item Is there a minimum? A maximum?
  \item Does it have an infinite chain?
  \item Does it have arbitrarily large antichains? That is, can you find an 
  antichain $A$ of size $|A| = k$ for every $k \in \N$?
\end{enumerate}
\end{exercise}

\begin{exerciseD} 
  Does every infinite subset $S \subseteq \nn$ contain
  an infinite chain?
\end{exerciseD}

\begin{exercise}
  Show that $(\nn,\leq)$ has no infinite antichain. \textbf{Hint.} Use 
  the previous exercise.
\end{exercise}



Consider the induced ordering on $\{0,1\}^n$. That is, for $x,y\in \{0,1\}^n$
we have $x \leq y$ if $x_i \leq y_i$ for every coordinate $i \in [n]$.

\begin{exercise}
 Draw the Hasse diagrams of $(\{0,1\}^n, \leq)$ for $n=2,3$.
\end{exercise}

\begin{exercise}
  Determine the maximum, minimum, maximal, and minimal elements of 
  $\{0,1\}^n$.
\end{exercise}

\begin{exercise}
  What is the longest chain of $\{0,1\}^n$?
\end{exercise}

\begin{exerciseDD}
  What is the largest antichain of $\{0,1\}^n$?
\end{exerciseDD}



\subsection{Infinite Sets}

In the lecture (and the lecture notes) we have showed that $\N \times \N \cong \N$, i.e.,
there is a bijection $f: \N \times \N \rightarrow \N$. From this, and by induction, it follows
quite easily that $\N^k \cong \N$ for every $k$.

\begin{exercise}
   Consider $\N^*$, the set of all finite sequences of natural numbers, that is,
   $\N^* = \{\epsilon\} \cup \N \cup \N^2 \cup \N^3 \cup \dots$. Here,
   $\epsilon$ is the empty sequence. Show that $\N \cong \N^*$ by defining
   a bijection $\N \rightarrow \N^*$.
\end{exercise}

\begin{exercise}
   Show that $R \cong R \times R$. \textbf{Hint:} Use the fact that 
   $R \cong \{0,1\}^{\N}$ and thus show that $\{0,1\}^{\N} \cong \{0,1\}^{\N} \times \{0,1\}^{\N}$.
\end{exercise}

\begin{exercise}
  Consider $\R^{\N}$, the set of all infinite sequences $(r_1, r_2, r_3,\dots)$ of real numbers.
  Show that $\R \cong \R^{\N}$. \textbf{Hint:} Again, use the fact that $\R \cong \{0,1\}^{\N}$.
\end{exercise}

Next, let us view $\{0,1\}^{\N}$ as a partial ordering: given two elements $\mathbf{a}, \mathbf{b} \in \{0,1\}^{\N}$,
that is, sequences $\mathbf{a} = (a_1,a_2,\dots)$ and $\mathbf{b} = (b_1,b_2,\dots)$, we define
$\mathbf{a} \leq \mathbf{b}$ if $a_i \leq b_i$ for all $i \in \N$. Clearly,
$(0,0,\dots)$ is the minimum element in this ordering and $(1,1,\dots)$ the maximum.\\

\begin{exercise}
   Give a countably infinite chain in $\{0,1\}^{\N}$. Remember that a set $A$ is countably infinite
   if $A \cong \N$.
\end{exercise}
    $$ (0, 0, 0, \dots) $$
    $$ (1, 0, 0, \dots) $$
    $$ (1, 1, 0, \dots) $$
    $$ (1, 1, 1, \dots) $$
    $$ \dots $$

    Since there are countably infinite bits in every element, we can construct countably infinite chain in $\{0,1\}^{\N}$ as showed above.

\begin{exercise}
   Find a countably infinite antichain in $\{0,1\}^{\N}$.
\end{exercise}
    $$ (1, 0, 0, \dots) $$
    $$ (0, 1, 0, \dots) $$
    $$ (0, 0, 1, \dots) $$
    $$ \dots $$

    Since there are countably infinite bits in every element, we can construct countably infinite chain in $\{0,1\}^{\N}$ as showed above.


\begin{exercise}
   Find an uncountable antichain in $\{0,1\}^{\N}$. That is, an antichain $A$ with $A \cong \R$.
\end{exercise}
    Since $\{0,1\}^{\N} \cong \R$, there is a bijection: $x \leftrightarrow \mathbf{t}$, $x \in \R, \mathbf{t} \in \{0,1\}^{\N}$. Let's consider $\mathbf{t_i}$.
    $$ \mathbf{t_i} = (a_1, a_2, \dots), a_k \in \{0,1\}, k \in \N $$
    Define $\bar{\mathbf{t_i}} = (1-a_1, 1-a_2, \dots)$.
    Then construct $\hat{\mathbf{t_i}}$ as:
    $$ \hat{\mathbf{t_i}} = (a_1, 1-a_1, a_2, 1-a_2, \dots) $$
    Consider $\hat{\mathbf{t_i}}, \hat{\mathbf{t_j}}, \forall i, j \in \N, i \neq j$.

    \textbf{Case 1:} If $\mathbf{t_i} \nleq \mathbf{t_j}$, obviously, $\hat{\mathbf{t_i}} \nleq \hat{\mathbf{t_j}}$.

    \textbf{Case 2:} If $\mathbf{t_i} \leq \mathbf{t_j}$
    
    $$\mathbf{t_i}=(a_1, a_2, \dots)\quad \bar{\mathbf{t_i}}=(1-a_1, 1-a_2, \dots)$$
    $$\mathbf{t_j} = (b_1, b_2, \dots)\quad \bar{\mathbf{t_j}} = (1-b_1, 1-b_2, \dots)$$

    According to the definition of $\mathbf{a} \leq \mathbf{b}$, we know that $a_k \leq b_k$. So, $\bar{\mathbf{t_i}} \geq \bar{\mathbf{t_j}}$.

    Compare every bit of $\hat{\mathbf{t}}$.
    $$
    \begin{array}{c|lcccr}
    \hat{\mathbf{t}} & 1 & 2 & 3 & 4 & \dots \\
    \hline
    \hat{\mathbf{t_i}} & a_1 & 1-a_1 & a_2 & 1-a_2 & \dots \\
    \hat{\mathbf{t_j}} & b_1 & 1-b_1 & b_2 & 1-b_2 & \dots \\
    \end{array}
    $$

    Since $a_k \leq b_k$, $1-a_k \geq 1-b_k$. 
    
    And since $i \neq j$, $\mathbf{t_i}, \mathbf{t_j}$ are not the same $\mathbf{t}$, which means that $\exists \eta, a_{\eta} < b_{\eta}, 1-a_{\eta} > 1-b_{\eta}$. So, $\hat{\mathbf{t_i}} \nleq \hat{\mathbf{t_j}}$.

    Therefore, $\hat{\mathbf{t_1}} \quad \hat{\mathbf{t_2}} \quad \dots$ is an uncountable antichain in $\{0,1\}^{\N}$.

\begin{exerciseDD}
   Find an uncountable chain in $\{0,1\}^{\N}$. That is, an antichain $A$ with $A \cong \R$.
\end{exerciseDD}






\end{document}



